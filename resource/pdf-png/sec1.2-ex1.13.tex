\documentclass[class=minimal,border=4pt,preview,12pt]{standalone}
\usepackage{amsmath,amssymb}
\begin{document}

for $\forall n > 1, \in \mathcal{N}$, we have:\\
$$f(n) = f(n-1) + f(n-2) $$
Suppose $f(n) = a x^n$, thus for $\forall n > 1$, we have:\\
\begin{align*}
ax^n & = ax^{n-1} + ax^{n-2} \\
\Leftrightarrow 0 &=x^2 - x - 1\\
\Leftrightarrow x &= \left\{
  \begin{array}{c}
     \frac{1 + \sqrt{5}}{2} \\
     \frac{1 - \sqrt{5}}{2}
  \end{array}\right.
\end{align*}
Thus, we have:\\
$$f(n) = a\left(\frac{1 + \sqrt{5}}{2}\right)^n + b\left(\frac{1 -
    \sqrt{5}}{2}\right)^n $$
As $f(0) = 0$ and $f(1) = 1$, we have:\\
$$a  = b = 1$$
Thus:\\
$$f(n) = \left(\frac{1 + \sqrt{5}}{2}\right)^n + \left(\frac{1 -
    \sqrt{5}}{2}\right)^n $$
\end{document}
